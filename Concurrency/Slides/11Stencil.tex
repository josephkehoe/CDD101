

\documentclass{beamer}
 
\usepackage[utf8]{inputenc}
 \usetheme{Madrid}
 \usecolortheme{beaver}
 \usefonttheme{structuresmallcapsserif}
 \usepackage{listings}
%Information to be included in the title page:


\title[The Stencil Pattern] %optional
{The Stencil Pattern}

\subtitle{A Short Introduction}

\author[Dr. Joseph Kehoe] % (optional, for multiple authors)
{Joseph Kehoe\inst{1}}

\institute[IT Carlow] % (optional)
{
	\inst{1}%
	Department of Computing and Networking\\
	Institute of Technology Carlow
}

\date[ITC 2017] % (optional)
{CDD101, 2017}

\logo{\includegraphics[height=1.5cm]{itcarlowlogo.png}}




 
 \AtBeginSection[]
 {
 	\begin{frame}
 		\frametitle{Table of Contents}
 		\tableofcontents[currentsection]
 	\end{frame}
 }
 
 
 
\begin{document}
 
\frame{\titlepage}
 
 
 
 \begin{frame}
 	\frametitle{Table of Contents}
 	\tableofcontents
 \end{frame}
 
 
 \section{Definition}
\begin{frame}
\frametitle{Definition}

\begin{itemize}
	\item The Stencil pattern is a special case of the Map pattern.
	\item The function is applied to every element in the set concurrently as with the map.
	\item But the input to the elemental function includes not just the current instance but also other elements in its local neighbourhood
	\item The output is a function of some neighbouring elements in an input collection.
	\item The function contains dependencies on its surrounding elements
\end{itemize}
\end{frame}

\begin{frame}
	\frametitle{Related Patterns}
\begin{itemize}
\item Stencil is a specific case of a Map

\item The map elemental functions has no dependencies but the stencil has

\item The stencil has read dependencies

\item It is a very common pattern in the real world

\item Can be one, two or N dimensional (usually represented as a grid (matrix)
\end{itemize}
\end{frame}
\section{Details}
\begin{frame}
	\frametitle{Details}
	\begin{itemize}
		\item Neighbourhood is specified using a set of fixed offsets relative to the output position
		\begin{itemize}
			\item Von Neuman neighbourhood (North, South, east and West)
			\item Moore Neighbourhood (8 compass points NE,NW,SE,SW,N,S,E,W)
		\end{itemize}
		\item Show these as offsets on 1,2 and 3D grids!
		\item Grid can be sparse or dense
		\begin{itemize}
			\item There are special ways of handling sparse matrices
		\end{itemize}
	\end{itemize}
\end{frame}

\section{Usage}
\begin{frame}
	\frametitle{Usage of Stencils}
\begin{itemize}
	\item Image and Signal Processing (medical, satelite and vision)
	\item Partial Differential Equation (PDE) solvers over regular grids
	\item Seismic Simulations (especially oil and gas)
	\item Weather Simulation
\end{itemize}

\end{frame}

\begin{frame}[fragile=singleslide]
	\frametitle{Serial Implementation}
\textbf{Basic Code}


\begin{lstlisting}
void stencil(int dim,float in[], float out[],
			func f,int size)
{
	float neighbours[size];
	for (int i=0; i < dim; ++i)
	{
		neighbours=calcNeighbours(in,i,size);
		out[i]=f(neighbours);
	}
}
\end{lstlisting}



\begin{itemize}
	\item calcNeighbours depends on neighbourhood type
\end{itemize}
\end{frame}






\section{Optimisation}
\begin{frame}
	\frametitle{Cache Behaviour}
\begin{itemize}
	\item Cache behaviour is very important
	\item The grid is usually divided into tiles
	\item Each thread computes all elements in one (or more) tile(s)
	\begin{itemize}
		\item Each tile is mostly self contained but elements at the edge of one tile will need to read elements in the adjacent tiles
		\item This is handled using \textbf{ghost} tiles that is tiles we need to read from another tile but do not update
		\item Set of ghost tiles known as \textbf{halo}
	\end{itemize}
	\item Each tile will need to communicate with adjacent tiles (why?)
	\item Shape of tiles will affect cache interaction
\end{itemize}
\end{frame}
 
 \begin{frame}
 	\frametitle{Tile Shape}
 	\begin{itemize}
 		\item Grid usually stored in array
 		\item When are two array locations stored beside each other in memory?
 		\begin{itemize}
 			\item Depends on programming language (Fortran vs C)
 		\end{itemize}
 		\item Optimising tiles for stencils sometimes known as \textbf{strip-mining}
 		 		\begin{itemize}
 		 			\item Strips are a multiple of cache line width (to prevent false sharing - what is this?)
 		 		\end{itemize}
 	\end{itemize}
 \end{frame}
\section{Recurrances}
\begin{frame}
\frametitle{Recurrances}
    \begin{itemize}
      \item Recurrances are neighbourhoods of \textbf{outputs}
      \item Instances in a recurrance can depend on values computed by other instances
      \item In serial code this appears as loop carried dependencies in which iterations of the loop depend on previous iterations
    \end{itemize}
Even though the loop iterations are not independent it is still possible to partallelize the loop
\end{frame}
\subsection{Example}
\begin{frame}
\frametitle{Example Recurrance}
\begin{itemize}
	\item Assume a grid (array) of data
	\item Each element a[i][k] depends on the output of the value to its left and above it
	\item That is, a[i][k] depends on a[i-1][k] and a[i][k-1]
	\item Show this on a diagram
	\item This can be paralellised using a \textbf{hyperplane}
	\item This can be difficult to implement
\end{itemize}
\end{frame}
\begin{frame}[fragile=singleslide]
	\frametitle{Example Recurrance}
\begin{lstlisting}
void recurrance(int xdim,int ydim, 
                float in[xdim][ydim], 
                float out[xdim][ydim])
{
  float neighbours[size];
  for (int i=0; i < xdim; ++i)
  {
    for(int k=0; k<ydim;++k)
    {
      out[i,k]=f(in[i,k],out[i-i][k],out[i][k-1]);
    }
  }
}
\end{lstlisting}
\end{frame}
\end{document}

