

\documentclass{beamer}
 
\usepackage[utf8]{inputenc}
 \usetheme{Madrid}
 \usecolortheme{beaver}
 \usefonttheme{structuresmallcapsserif}
 \usepackage{listings}
%Information to be included in the title page:


\title[Concurrency] %optional
{Semaphores}

\subtitle{An Overview}

\author[Dr. Joseph Kehoe] % (optional, for multiple authors)
{Joseph Kehoe\inst{1}}

\institute[IT Carlow] % (optional)
{
	\inst{1}%
	Department of Computing and Networking\\
	Institute of Technology Carlow
}

\date[ITC 2017] % (optional)
{CDD101, 2017}

\logo{\includegraphics[height=1.5cm]{../../itcarlowlogo.png}}




 
 \AtBeginSection[]
 {
 	\begin{frame}
 		\frametitle{Table of Contents}
 		\tableofcontents[currentsection]
 	\end{frame}
 }
 
 
 
\begin{document}
 
\frame{\titlepage}
 
 

\begin{frame}
\frametitle{What is a Semaphore?}
Invented by Edsgar Dijkstra
\begin{itemize}
\item A Semaphore is an integer with the following properties
\begin{enumerate}
\item When you create a semaphore you can initialise it to any integer value but after that you can only perform two operations on it. 
\item You can increment by one or decrament it by one.
\item You cannot read the current value of a semaphore
\item When a thread decrements a semaphore, if the result is negative, the thread blocks itself and cannot continue until the semaphore value is no longer negative
\item When a thread increments a semaphore, if there are other threads waiting on that semaphore  then one of them becomes unblocked
\end{enumerate}
\end{itemize}
\end{frame}


\begin{frame}
\frametitle{Consequences of Definition}


\begin{itemize}
\item When you signal a semaphore you do not necessarily know whether another thread is waiting
\begin{itemize}
 \item so the number of unblocked threads may be zero or one
 \end{itemize} 
\item In general there is no way of knowing whether a thread will block on a decrement operation
\item After an increment operation both the incrementing thread and one waiting thread can run concurrently
\begin{itemize}
 \item but there is no way of knowing which (if either) will continue immediately
 \end{itemize}
\end{itemize}
\end{frame}


\begin{frame}
	\frametitle{Consequences}
  The value of a semaphore indicates:
\begin{itemize}
	\item Positive integer represents the number of threads that can decrement without blocking
\item Negative integer represents the number of waiting (blocked) threads
\item Zero means no threads are waiting
\end{itemize}
But you are not allowed to ask a semaphore what its value is!
\end{frame}


\begin{frame}
	\frametitle{Consequences}
  Only two operations are allowed on a semaphore:
\begin{enumerate}
	\item Decrement the counter
		\begin{itemize}
		\item Called dec(), wait() or P()
		\end{itemize}
	\item Increment the counter
		\begin{itemize}
		\item Called inc(), signal() or V()
		\end{itemize}
\end{enumerate}
But (again!) you are not allowed to ask a semaphore what its value is!
\end{frame}


\begin{frame}
	\frametitle{Why use Semaphores}
\begin{itemize}
	\item They impose constraints that help programmers avoid errors
	\item Code using semaphores tends to be  clean and organised
	\item Semaphores have efficient implementations
	\item Mainly we use them to force you to think clearly about the issues of concurrency
\end{itemize}
	Lessons learned here will be applicable to any concurrency programming model you use in the future
\end{frame}

\begin{frame}
	\frametitle{Creating Semaphores in C++}
	Look at the following files online:
	\begin{itemize}
	\item Semaphore.h
	\item Semaphore.cpp
	\end{itemize}
\end{frame}

\begin{frame}
	\frametitle{Signalling with Semaphores}
	\begin{itemize}
		\item A single semaphore can be used to send a signal from one thread to another
		\item To indicate something has happened
		\item Use a semaphore initialised with value 0
\begin{itemize}
\item Thread waiting for signal calls wait
\item Thread sending signal calls signal
\end{itemize}
\end{itemize}

\end{frame}

\begin{frame}
	\frametitle{Simple Rendezvous}
	Generalised Signal Pattern
	\begin{itemize}
	\item Thread 1 has to wait for thread 2
	\item Thread 2 has to wait for thread 1
	\item Both have to arrive at a certain point before either proceeds
	\item Thread A
	\begin{itemize}
		\item A1;
		\item A2;
	\end{itemize}
	\item Thread B
	\begin{itemize}
		\item B1;
		\item B2;
	\end{itemize}
	\item A1 must finish before B2 starts
	\item B1 must finish before A2 starts
	\end{itemize}
\end{frame}


\begin{frame}
	\frametitle{Exercise}
	\begin{itemize}
	\item Implement both solutions in C++
	\item Create Makefile
	\item Document with Doxygen
	\item Put solution up on github
	\end{itemize}
	
\end{frame}


    
\end{document}

