

\documentclass{beamer}
 
\usepackage[utf8]{inputenc}
 \usetheme{Madrid}
 \usecolortheme{beaver}
 \usefonttheme{structuresmallcapsserif}
 \usepackage{listings}
%Information to be included in the title page:


\title[Reduce and Scan] %optional
{Reduce and Scan Patterns}

\subtitle{A Short Introduction}

\author[Dr. Joseph Kehoe] % (optional, for multiple authors)
{Joseph Kehoe\inst{1}}

\institute[IT Carlow] % (optional)
{
	\inst{1}%
	Department of Computing and Networking\\
	Institute of Technology Carlow
}

\date[ITC 2017] % (optional)
{CDD101, 2017}

\logo{\includegraphics[height=1.5cm]{itcarlowlogo.png}}




 
 \AtBeginSection[]
 {
 	\begin{frame}
 		\frametitle{Table of Contents}
 		\tableofcontents[currentsection]
 	\end{frame}
 }
 
 
 
\begin{document}
 
\frame{\titlepage}
 
 
 
 \begin{frame}
 	\frametitle{Table of Contents}
 	\tableofcontents
 \end{frame}
 
 
 \section{Reduce}
\begin{frame}
\frametitle{Definition}

\begin{itemize}
	\item The Reduce pattern is where a sequence of input data is reduced down to a single output value.
	\item A Combiner function is applied to every member of the input set.
	\item The combiner function operates a a pair of input values \textbf{result=combine(a,b)}
	\item Examples include using \textbf{+} to get the sum of a sequence
	\item We assume that the associated pairs of input can be combined in different orders and still get the same answers
	\item e.g.  a+b+c = b+a+c = b+c+a = a+c+b = c+a+b = c+b+a
\end{itemize}
\end{frame}


 \begin{frame}[fragile=singleslide]
 	\frametitle{Simple Example}
\begin{lstlisting}
 	float sum(int dim,float in[])
 	{
	 	float sum=0.0;
 		for (int i=0; i < dim; ++i)
 		{
 			sum +=in[i];
 		}
 		return sum;
 	}
 \end{lstlisting}
 \end{frame}
  \begin{frame}[fragile=singleslide]
  	\frametitle{Simple OpenMP Implementation}
  	Reduction is built in for simple operators
  	\begin{lstlisting}
  	float sum(int dim,float in[])
  	{
  	float sum=0.0;
  	#pragma omp simd parallel for reduction(+:sum)
  	for (int i=0; i < dim; ++i)
  	{
  	sum +=in[i];
  	}
  	return sum;
  	}
  	\end{lstlisting}
  \end{frame}
  \section{Complex Operators}
\begin{frame}
	\frametitle{More Complex Operators}
\begin{itemize}
\item What if our operator is not supported by the \textbf{reduction} clause?

\item Produce our own code using Tiling

\item First each thread produces a result for its own subsequence

\item Then combine all results for each tile into one value
\end{itemize}
\end{frame}
  \begin{frame}[fragile=singleslide]
  	\frametitle{Simple OpenMP Implementation}
  	\begin{lstlisting}
  	float sum(int dim,float in[])
  	{
  	float result=0.0;
  	float tileResult[NumThreads];
  	#pragma omp parallel for
  	for (int i=0; i < dim; ++i)
  	{
  	  int tid = omp_get_thread_num();
  	  tileResult[tid] =op(tileResult[tid],in[i]);
  	}
  	for (int i=0; i < NumThreads; ++i)
  	{
  	  result=op(tileResult[i],result);
  	}
  	return result;
  	}
  	\end{lstlisting}
  \end{frame}
  
  \section{Scan}
  \begin{frame}
  	\frametitle{Definition}
  	
  	\begin{itemize}
  		\item Produces all partial reductions of an input sequence to produce a new output sequence
  		\item used in e.g. integration
  		\item Using sum as an example - each element will contain the sun of all previous elements
  		\begin{itemize}
  			\item \textbf{Inclusive} scan means the sum includes everything up to and including the current element
  			\item \textbf{Exclusive} scan means we sum only the previous elements (do not include ourselves)
  			\item Show output of each type on sequence: 3, 4, 6, 8, 1, 4
  		\end{itemize}
  	\end{itemize}
  \end{frame}
   \begin{frame}
   	\frametitle{Approach}
   	If using standard fork join we use a three phase approach
   	\begin{enumerate}
   		\item Tile the sequence and reduce each tile in parallel
   		\item Do an eclusive scan of the reduction values (always exclusive)
   		\item Scan each of the tiles where the initial value is that calculated by the previous step
   	\end{enumerate}
   	Or use \textbf{tasks} to implement a tree based approach
   \end{frame}
\end{document}

