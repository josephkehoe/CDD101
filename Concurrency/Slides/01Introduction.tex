

\documentclass{beamer}
 
\usepackage[utf8]{inputenc}
 \usetheme{Madrid}
 \usecolortheme{beaver}
 \usefonttheme{structuresmallcapsserif}
 \usepackage{listings}
%Information to be included in the title page:


\title[Concurrency] %optional
{Concurrency}

\subtitle{A Short Introduction}

\author[Dr. Joseph Kehoe] % (optional, for multiple authors)
{Joseph Kehoe\inst{1}}

\institute[IT Carlow] % (optional)
{
	\inst{1}%
	Department of Computing and Networking\\
	Institute of Technology Carlow
}

\date[ITC 2017] % (optional)
{CDD101, 2017}

\logo{\includegraphics[height=1.5cm]{../../itcarlowlogo.png}}




 
 \AtBeginSection[]
 {
 	\begin{frame}
 		\frametitle{Table of Contents}
 		\tableofcontents[currentsection]
 	\end{frame}
 }
 
 
 
\begin{document}
 
\frame{\titlepage}
 
 
 
 \begin{frame}
 	\frametitle{Table of Contents}
 	\tableofcontents
 \end{frame}
 
 
 \section{Why Bother?}
\begin{frame}
\frametitle{Processor Hard Limits}
Moore’s Law is in trouble
\begin{itemize}
\item Power Wall
\begin{itemize}
\item Heat dissipation
\end{itemize}
\item Memory Bottleneck
\begin{itemize}
\item Von Neumann Architecture
\end{itemize}
\item Physical Size Limits
\begin{itemize}
\item Cannot go much smaller
\end{itemize}
\item Complexity Issues
\begin{itemize}
\item Pipelining, Lookahead, Out of Order Execution, Instruction Level Parallelism
\end{itemize}
\end{itemize}
\end{frame}

\begin{frame}
	\frametitle{New Architectures are Required}
\begin{enumerate}
	\item Multicores
	\item General Purpose GPUs
	\item Clusters – the new supercomputer architecture
	\item Manycore (Xeon Phi)
	\end{enumerate}	
Each has its own issues to overcome

Each requires a different programming approach

ALL REQUIRE PARALLELISM
	
\end{frame}
\section{Definitions}
\begin{frame}
	\frametitle{Concurrency versus Parallelism}
	\begin{itemize}
	\item Concurrency  is the decomposability property of a program, algorithm, or problem into order-independent or partially-ordered components or units
\item Parallellism is a type of computation in which many calculations or the execution of processes are carried out simultaneously

	\end{itemize}
All parallel programs are concurrent but not all concurrent programs are parallel
\end{frame}

\begin{frame}
	\frametitle{Processes and Threads}
	\begin{itemize}
	\item Both processes and threads are independent sequences of execution. 
	\item The typical difference is that threads (of the same process) run in a shared memory space, while processes run in separate memory spaces.
\item Processes communicate via message passing commonly termed Inter Process Communication (IPC) (using e.g. TCP/IP)
\item As threads share memory space they can access each others memory (and therefore variables) directly commonly called the Shared Memory Approach
\begin{itemize}
\item See also hardware threads, hyper threading and SMT
\end{itemize}

	\end{itemize}
	
\end{frame}

\section{Issues}
\begin{frame}
	\frametitle{Issues}
\begin{itemize}
	\item Sequential Tools and Thinking
	\item Memory Bottlenecks
	\item No single model of Parallel Architecture yet
	\item Load Balancing
	\item Non determinism
	\item New Problems
\begin{itemize}
	\item Mutual Exclusion 
	\end{itemize}
	\item New categories of Error
	\begin{itemize}
	\item Deadlock
		\end{itemize}
	\end{itemize}	
	
\end{frame}


\begin{frame}
	\frametitle{What we want (need)!}
	\begin{itemize}
	\item Scalability
	\item Speedup
	\item Efficiency
	\item Portability
	\item Maintainability
	\item Determinism
	\item Composability
	\item Safety
	\item Ease of Development
	\end{itemize}
	
\end{frame}

    
\end{document}

