

\documentclass{beamer}
 
\usepackage[utf8]{inputenc}
 \usetheme{Madrid}
 \usecolortheme{beaver}
 \usefonttheme{structuresmallcapsserif}
 \usepackage{listings}
%Information to be included in the title page:


\title[Concurrency] %optional
{Some Classic Concurrency Problems}

\subtitle{Reusable Barriers}

\author[Dr. Joseph Kehoe] % (optional, for multiple authors)
{Joseph Kehoe\inst{1}}

\institute[IT Carlow] % (optional)
{
	\inst{1}%
	Department of Computing and Networking\\
	Institute of Technology Carlow
}

\date[ITC 2017] % (optional)
{CDD101, 2017}

\logo{\includegraphics[height=1.5cm]{../../itcarlowlogo.png}}




 
 \AtBeginSection[]
 {
 	\begin{frame}
 		\frametitle{Table of Contents}
 		\tableofcontents[currentsection]
 	\end{frame}
 }
 
 
 
\begin{document}
 
\frame{\titlepage}
 
 

\begin{frame}
\frametitle{The Problem}
\begin{itemize}
\item The previous barrier solution will not work in a loop
	\begin{itemize}
	\item Why?
	\end{itemize}
\item How do we solve this problem?
\item We need a reusable barrier that locks itself after all the threads have passed through
\item Try it now!
\end{itemize}
\end{frame}


\begin{frame}[fragile]
\frametitle{Why is this a bad solution?}
\begin{verbatim}
mutex.wait ()
    count += 1
mutex.signal ()

if count == n : turnstile.signal ()
turnstile.wait ()
turnstile.signal ()
//CRITICAL POINT

mutex.wait ()
    count -= 1
mutex.signal ()
if count == 0: turnstile.wait ()
\end{verbatim}
\end{frame}


\begin{frame}[fragile]
	\frametitle{Barrier Hint}
\begin{verbatim}
turnstile = Semaphore (0)
turnstile2 = Semaphore (1)
mutex = Semaphore (1)
\end{verbatim}
\begin{itemize}
	\item Use two turnstiles
\item One turnstile is always locked!
\item Sometimes called a two phase barrier
\end{itemize}
\end{frame}



\end{document}

