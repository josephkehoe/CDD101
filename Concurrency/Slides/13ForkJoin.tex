

\documentclass{beamer}
 
\usepackage[utf8]{inputenc}
 \usetheme{Madrid}
 \usecolortheme{beaver}
 \usefonttheme{structuresmallcapsserif}
 \usepackage{listings}
%Information to be included in the title page:


\title[Reduce and Scan] %optional
{Fork - Join Pattern}

\subtitle{A Short Introduction}

\author[Dr. Joseph Kehoe] % (optional, for multiple authors)
{Joseph Kehoe\inst{1}}

\institute[IT Carlow] % (optional)
{
	\inst{1}%
	Department of Computing and Networking\\
	Institute of Technology Carlow
}

\date[ITC 2017] % (optional)
{CDD101, 2017}

\logo{\includegraphics[height=1.5cm]{itcarlowlogo.png}}




 
 \AtBeginSection[]
 {
 	\begin{frame}
 		\frametitle{Table of Contents}
 		\tableofcontents[currentsection]
 	\end{frame}
 }
 
 
 
\begin{document}
 
\frame{\titlepage}
 
 
 
 \begin{frame}
 	\frametitle{Table of Contents}
 	\tableofcontents
 \end{frame}
 
 
 \section{Overview}
\begin{frame}
\frametitle{Definition}

\begin{itemize}
	\item Pattern used to implement Map, Reduce, Recurrance and Scan (amongst others)
	\item Uses a Divide and Conquer Approach
	\item Recursively break a problem up into subproblems until base case is reached
	\item Base Case is amount of work too small to be parallelised further (problem dependent)
	\item It is important that the subproblems are independent
\end{itemize}
\end{frame}


 \begin{frame}[fragile=singleslide]
 	\frametitle{Pseudo-code}
\begin{lstlisting}
 	void divide&Conquer(P)
 	{
	 	if (P is BASE CASE){
	 	  solve P;
	 	} else {
	 	  Divide P into K subproblems
	 	  foreach  P[i]:0<=i<K{
	 	    fork(divide&Conquer(P[i]))
	 	  }
	 	  join
	 	  Combine all partial solutions P[i] 
	 	}
	 }
 \end{lstlisting}
 \end{frame}
 
\begin{frame}
	\frametitle{Speed}
\begin{itemize}
\item if there are K divisions of the problem at each recursive level
\item and there are N levels of recursion
\item then we get $K^{N}$-way parallelism
\item getting the no. of levels $N$ correct is important (and problem dependent)
\end{itemize}
\end{frame}

\section{OpenMP}
  \begin{frame}[fragile=singleslide]
  	\frametitle{Simple OpenMP Implementation}
  	\begin{lstlisting}
#pragma omp task
fnA();
fnB();
#pragma omp taskwait
  	\end{lstlisting}
  \end{frame}
  
  \begin{frame}
  	\frametitle{Notes}
  	
  	\begin{itemize}
  		\item taskwait acts as join point or "barrier"
  		\item task causes next line of code to run in parallel
  		\item We can use \{\} to make block of code run in parallel
  		\item Must have pragma parallel in code before task otherwise it will not run in parallel (explain!)
  		\item Global variables captured by reference (shared)
  		\item Local variables captured by value (firstprivate)
  	\end{itemize}
  \end{frame}

\end{document}

