

\documentclass{beamer}
 
\usepackage[utf8]{inputenc}
 \usetheme{Madrid}
 \usecolortheme{beaver}
 \usefonttheme{structuresmallcapsserif}
 \usepackage{listings}
%Information to be included in the title page:


\title[Java RMI] %optional
{Java RMI}

\subtitle{An Overview}

\author[Dr. Joseph Kehoe] % (optional, for multiple authors)
{Joseph Kehoe\inst{1}}

\institute[IT Carlow] % (optional)
{
	\inst{1}%
	Department of Computing and Networking\\
	Institute of Technology Carlow
}

\date[ITC 2018] % (optional)
{CDD101, 2018}

\logo{\includegraphics[height=1.5cm]{../../itcarlowlogo.png}}




 
 \AtBeginSection[]
 {
 	\begin{frame}
 		\frametitle{Table of Contents}
 		\tableofcontents[currentsection]
 	\end{frame}
 }
 
 
 
\begin{document}
 
\frame{\titlepage}
 
%  \begin{frame}
%  	\frametitle{Table of Contents}
%  	\tableofcontents
%  \end{frame}
 


  \begin{frame}
  	\frametitle{Introduction}

  	\begin{itemize}
  		\item RMI is Remote Method Invocation
  		\item RPC for Objects
  		\item Integrated into the Java Language
  		\item Easy to use
  		\item allows code to call objects in another JVM
  		\item Runs over TCP/IP
  	\end{itemize}
  \end{frame}
  
    \begin{frame}
    	\frametitle{Java RMI vs. Corba}
    	\begin{itemize}
    		\item Integrated directly into Java Language
    		\item Easier to use than Corba
    		\item Less overhead (marshalling and unmarshalling etc.)
    		\item Corba allows different languages to talk to each other transparently
    		\item RMI requires JVM on server - Corba is platform independent
    		\item Corba is more powerful (services such as object migration)
    	\end{itemize}
    \end{frame}
 
 
     \begin{frame}
     	\frametitle{RMI-IIOP}
     	\begin{itemize}
     		\item RMI over IIOP (Internet Inter-Orb Protocol)
     		\item Allows Java RMI to access Corba using the Java Remote Method Protocol (JRMP) as the transport	
     		\item Simplifies access to Corba but still gets all the benefits
     		\item Reduces complexity and footprint of Corba
     	\end{itemize}

     \end{frame}
     
      
 
    \begin{frame}
    	\frametitle{Sample Interface Code}

    	\begin{example}
    	import java.rmi.Remote;
    	
    	import java.rmi.RemoteException;
    	
    	public interface RmiServerIntf extends Remote \{
    	
    	public String getMessage() throws RemoteException;
    	
    	\}
        \end{example}	
    \end{frame}           
    \begin{frame}
    	\frametitle{Sample Client Code}

    	\begin{example}
import java.rmi.Naming;

public class RmiClient \{
 
	public static void main(String args[]) throws Exception \{

		RmiServerIntf obj = (RmiServerIntf)Naming.lookup("//localhost/RmiServer");

		System.out.println(obj.getMessage()); 

	\}

\}
    	\end{example}	
    \end{frame}            
    \begin{frame}
    	\frametitle{Sample Server Code}

    	\begin{example}
import java.rmi.Naming;

import java.rmi.RemoteException;

import java.rmi.server.UnicastRemoteObject;

import java.rmi.registry.*; 

public class RmiServer extends UnicastRemoteObject implements RmiServerIntf \{

	public static final String MESSAGE = "Hello World";
	
	public RmiServer() throws RemoteException \{
	
		super(0);    // required to avoid the 'rmic' step, see below

	\}
	
	public String getMessage() \{

		return MESSAGE;

	\}
    	\end{example}	
    \end{frame}  
    \begin{frame}
    	\frametitle{Sample Server Code}
    	
    	\begin{example}
    public static void main(String args[]) throws Exception \{
    
    	System.out.println("RMI server started");
    	
    	try \{ //special exception handler for registry creation
    		
    		LocateRegistry.createRegistry(1099); 
    		
    		System.out.println("java RMI registry created.");
    		
    	\} catch (RemoteException e) \{
    	//do nothing, error means registry already exists
    	
    	System.out.println("java RMI registry already exists.");

    \}
    
    //Instantiate RmiServer
    
    RmiServer obj = new RmiServer();
    
    // Bind this object instance to the name "RmiServer"

    Naming.rebind("//localhost/RmiServer", obj);

    System.out.println("PeerServer bound in registry");
    
\}
\}
    	\end{example}	
    \end{frame}  
        \begin{frame}
        	\frametitle{Sample Server Code}
Try the example \href{https://docs.oracle.com/javase/tutorial/rmi/overview.html}{here!}        	
	
        \end{frame} 
        
\end{document}

